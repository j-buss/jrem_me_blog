\documentclass{article}

\usepackage{amsmath}

\title{Chapter 1 - Introduction}

\begin{document}
\maketitle



\subsection*{Theorem 1.1 [2-steps]}
If an operation consists of two steps, of which the first can be done in \(n_1\) ways and for each of these the second can be done in \(n_2\) ways, then the whole operation can be done in \(n_1 \cdot n_2\) ways.

\subsection*{Theorem 1.2 [k-steps]}
If an operation consists of \(k\) steps, of which the first can be done in \(n_1\) ways, for each of these the second step can be done in \(n_2\) ways, for each of the first two the third step can be done in \(n_3\) ways, and so forth, then the whole operation can be done in \(n_1 \cdot n_2 \cdot \ldots \cdot n_k\) ways.

\subsection*{Theorem 1.3 [\# of permutations]} The number of permutations of \(n\) distinct objects is \(n!\)

\subsection*{Theorem 1.4 [\# of permutations, \(r\) at a time]} The number of permutations of \(n\) distinct objects taken \(r\) at a time is
\[{}_nP_r = \frac{n!}{(n-r)!}\]
for \(r=0,1,2,\ldots,n\).

\subsubsection*{Proof} The formula \({}_nP_r=n(n-1) \cdot \ldots \cdot (n-r+1)\) cannot be used for \(r=0\), but we do have

\[{}_nP_0 = \frac{n!}{(n-0)!}=1\]

For \(r=1,2,\ldots,n,\) we have


\begin{align}
{}_nP_r&=n(n-1)(n-2) \cdot \ldots \cdot (n-r-1)\\
&=\frac{n(n-1)(n-2) \cdot \ldots \cdot (n-r-1)(n-r)!}{(n-r)!}\\
&=\frac{n!}{(n-r)!}
\end{align}


\subsection*{Theorem 1.5 [Circular Permutations]} The number of permutations of \(n\) distinct objects arranged in a circle is \((n-1)!\)

\subsection*{Theorem 1.6 [Permutations of \(n\) objects]} The number of permutations of \(n\) objects of which \(n_1\) are of one kind, \(n_2\) are of a second kind, \(\ldots, n_k\) are of a \(k\)th kind, and \(n_1+n_2+\dots+n_k=n\) is

\[\frac{n!}{n_1! \cdot n_2! \cdot \ldots \cdot n_k!}\]



\end{document}