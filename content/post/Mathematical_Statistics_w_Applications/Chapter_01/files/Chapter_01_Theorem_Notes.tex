\documentclass{article}

\usepackage{amsmath}

\title{Chapter 1 - Introduction}

\begin{document}
\maketitle



\subsection*{Theorem 1.1 [2-steps]}
If an operation consists of two steps, of which the first can be done in \(n_1\) ways and for each of these the second can be done in \(n_2\) ways, then the whole operation can be done in \(n_1 \cdot n_2\) ways.

\subsection*{Theorem 1.2 [k-steps]}
If an operation consists of \(k\) steps, of which the first can be done in \(n_1\) ways, for each of these the second step can be done in \(n_2\) ways, for each of the first two the third step can be done in \(n_3\) ways, and so forth, then the whole operation can be done in \(n_1 \cdot n_2 \cdot \ldots \cdot n_k\) ways.

\subsection*{Theorem 1.3 [\# of permutations]} The number of permutations of \(n\) distinct objects is \(n!\)

\subsection*{Theorem 1.4 [\# of permutations, \(r\) at a time]} The number of permutations of \(n\) distinct objects taken \(r\) at a time is
\[{}_nP_r = \frac{n!}{(n-r)!}\]
for \(r=0,1,2,\ldots,n\).

\subsubsection*{Proof} The formula \({}_nP_r=n(n-1) \cdot \ldots \cdot (n-r+1)\) cannot be used for \(r=0\), but we do have

\[{}_nP_0 = \frac{n!}{(n-0)!}=1\]

For \(r=1,2,\ldots,n,\) we have


\begin{align*}
{}_nP_r&=n(n-1)(n-2) \cdot \ldots \cdot (n-r-1)\\
&=\frac{n(n-1)(n-2) \cdot \ldots \cdot (n-r-1)(n-r)!}{(n-r)!}\\
&=\frac{n!}{(n-r)!}
\end{align*}


\subsection*{Theorem 1.5 [Circular Permutations]} The number of permutations of \(n\) distinct objects arranged in a circle is \((n-1)!\)

\subsection*{Theorem 1.6 [Permutations of \(n\) objects]} The number of permutations of \(n\) objects of which \(n_1\) are of one kind, \(n_2\) are of a second kind, \(\ldots, n_k\) are of a \(k\)th kind, and \(n_1+n_2+\dots+n_k=n\) is

\[\frac{n!}{n_1! \cdot n_2! \cdot \ldots \cdot n_k!}\]

\subsection*{Theorem 1.7 [Combination]} The number of combinations of \(n\) distinct objects taken \(r\) at a time is 

\[\dbinom n r = \frac{n!}{r!(n-r)!}\]

for \(r=0,1,2,\ldots,n\)

\subsection*{Theorem 1.8 [\(n\) objects into \(k\) subsets]} The number of ways in which a set of \(n\) distinct objects can be partitioned into \(k\) subsets with \(n_1\) objects in the first subset, \(n_2\) objects in the second subset, \(\ldots\), and \(n_k\) objects in the \(k\)th subset is 

\[ \dbinom {n} {n_1, n_2, \ldots, n_k}=\frac{n!}{n_1! \cdot n_2! \cdot \ldots \cdot n_k!} \]

\subsubsection*{Proof} Since the \(n_1\) objects going into the first subset can be chosen in \(\dbinom n {n_1} \) ways, the \(n_2\) objects going into the second subset can then be chose in \( \dbinom {n-n_1} {n_2}\) ways, the \(n_3\) objects going into the third subset can then be chosen in \( \dbinom {n-n_1-n_2} {n_3}\) ways, and so forth, it follows by Theorem 1.2 that the total number of partitions is

\begin{align*}
\dbinom {n} {n_1,n_2,\ldots,n_k} &= \dbinom n {n_1} \cdot \dbinom {n - n_1} {n_2} \cdot \ldots \cdot \dbinom {n-n_1-n_2-\cdots -n_{k-1}}{n_k}\\
&= \frac{n!}{n_1! \cdot (n-n_1)!} \cdot \frac{(n-n_1)!}{n_2! \cdot (n-n_1-n_2)!} \cdot \ldots \cdot \frac{(n-n_1-n_2-\cdots-n_{k-1})!}{n_k! \cdot 0!}\\
&= \frac{n!}{n_1! \cdot n_2! \cdot \ldots \cdot n_k!}
\end{align*}


\subsection*{Theorem 1.9 [Binomial Coefficient]} \((x+y)^n=\sum^n_{r=0} \dbinom n r x^{n-r} y ^r\) for any positive integer \(n\)

\subsection*{Theorem 1.10 [Combination of the complimentary set]} For any positive integers \(n\) and \(r=0,1,2,\ldots,n,\)

\[\dbinom n r = \dbinom n {n-r}\]

\subsubsection*{Proof} We might argue that when we select a subset of \(r\) objects from a set of \(n\) distinct objects, we leave a subset of \(n-r\) objects; hence, there are as many ways of selecting \(r\) objects as there are ways of leaving (or selecting) \(n-r\) objects. To prove the theorem algebraically, we write

\begin{align*}
\dbinom n {n-r} &= \frac{n!}{(n-r)![n-(n-r)]!} = \frac{n!}{(n-r)!r!}\\
&= \frac{n!}{r!(n-r)!} = \dbinom n r
\end{align*}

\subsection*{Theorem 1.11 [Combination for Pascal's Triangle]} For any positive integer \(n\) and \(r=1,2,\ldots, n-1\),

\[\dbinom n r = \dbinom {n-1} r + \dbinom {n-1} {r-1}\]

\subsubsection*{Proof} Substituting \(x=1\) into \((x+y)^n\), let us write \((1+y)^n = (1+y)(1+y)^{n-1}=(1+y)^{n-1}+y(1+y)^{n-1}\) and equate the coefficient of \(y^r\) in \((1+y)^{n}\) with that in \((1+y)^{n-1}+y(1+y)^{n-1}\). Since the coefficient of \(y^r\) in \((1+y)^{n}\) is \(\dbinom n r \) and the coefficient of \(y^r\) in \((1+y)^{n-1}+(1+y)^{n-1}\) is the sum of the coefficient of \(y^r\) in \((1+y)^{n-1}\), that is, \(\dbinom {n-1} r\), and the coefficient of \(y^{r-1}\) in \((1+y)^{n-1}\), that is, \(\dbinom {n-1} {r-1} \), we obtain 

\[\dbinom n r = \dbinom {n-1} r + \dbinom {n-1} {r-1}\]

which completes the proof.

\subsection*{Theorem 1.12 [Sums of combinations]}

\[
\sum^k_{r=0} \dbinom m k \dbinom n {k-r} = \dbinom {m+n} k 
\]

\subsubsection*{Proof} Using the same technique as in the proof of Theorem 1.11, let us prove this theorem by equating the coefficients of \(y^k\) in the expressions on both sides of the equation

\[(1+y)^{m+n}=(1+y)^m(1+y)^n\]

The coefficient of \(y^k\) in \((1+y)^{m+n}\) is \(\dbinom {m+n} k\), and the coefficient of \(y^k\) in 

\[(1+y)^m(1+y)^n = [ \dbinom m 0 + \dbinom m 1 y + \cdots + \dbinom m m y^m ] \times [ \dbinom n 0 + \dbinom n 1 y + \cdots + \dbinom n n y^n ] \]

is the sum of the prodcts that we obtain by multiplying the constant term of the first factor by the coefficient of \(y^k\) in the second factor, the coefficient of \(y\) in the first factor by the coefficient of \(y^{k-1}\) in the second factor, ..., and the coefficient of \(y^k\) in the first factor by the constant term of the second factor. Thus, the coefficient of \(y^k\) in \((1+y)^m(1+y)^n\) is

\[
\dbinom m 0 \dbinom n k + \dbinom m 1 \dbinom n {k-1} + \dbinom m 2 \dbinom n {n-2} + \cdots + \dbinom m k \dbinom n 0 = \sum^k_{r=0} \dbinom m r \dbinom n {k-r}
\]

and this completes the proof.

\end{document}